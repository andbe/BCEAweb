% Options for packages loaded elsewhere
\PassOptionsToPackage{unicode}{hyperref}
\PassOptionsToPackage{hyphens}{url}
%
\documentclass[
]{article}
\usepackage{amsmath,amssymb}
\usepackage{iftex}
\ifPDFTeX
  \usepackage[T1]{fontenc}
  \usepackage[utf8]{inputenc}
  \usepackage{textcomp} % provide euro and other symbols
\else % if luatex or xetex
  \usepackage{unicode-math} % this also loads fontspec
  \defaultfontfeatures{Scale=MatchLowercase}
  \defaultfontfeatures[\rmfamily]{Ligatures=TeX,Scale=1}
\fi
\usepackage{lmodern}
\ifPDFTeX\else
  % xetex/luatex font selection
\fi
% Use upquote if available, for straight quotes in verbatim environments
\IfFileExists{upquote.sty}{\usepackage{upquote}}{}
\IfFileExists{microtype.sty}{% use microtype if available
  \usepackage[]{microtype}
  \UseMicrotypeSet[protrusion]{basicmath} % disable protrusion for tt fonts
}{}
\makeatletter
\@ifundefined{KOMAClassName}{% if non-KOMA class
  \IfFileExists{parskip.sty}{%
    \usepackage{parskip}
  }{% else
    \setlength{\parindent}{0pt}
    \setlength{\parskip}{6pt plus 2pt minus 1pt}}
}{% if KOMA class
  \KOMAoptions{parskip=half}}
\makeatother
\usepackage{xcolor}
\usepackage[margin=1in]{geometry}
\usepackage{graphicx}
\makeatletter
\def\maxwidth{\ifdim\Gin@nat@width>\linewidth\linewidth\else\Gin@nat@width\fi}
\def\maxheight{\ifdim\Gin@nat@height>\textheight\textheight\else\Gin@nat@height\fi}
\makeatother
% Scale images if necessary, so that they will not overflow the page
% margins by default, and it is still possible to overwrite the defaults
% using explicit options in \includegraphics[width, height, ...]{}
\setkeys{Gin}{width=\maxwidth,height=\maxheight,keepaspectratio}
% Set default figure placement to htbp
\makeatletter
\def\fps@figure{htbp}
\makeatother
\setlength{\emergencystretch}{3em} % prevent overfull lines
\providecommand{\tightlist}{%
  \setlength{\itemsep}{0pt}\setlength{\parskip}{0pt}}
\setcounter{secnumdepth}{-\maxdimen} % remove section numbering
\usepackage{graphicx} \usepackage{bm}
\ifLuaTeX
  \usepackage{selnolig}  % disable illegal ligatures
\fi
\IfFileExists{bookmark.sty}{\usepackage{bookmark}}{\usepackage{hyperref}}
\IfFileExists{xurl.sty}{\usepackage{xurl}}{} % add URL line breaks if available
\urlstyle{same}
\hypersetup{
  pdftitle={Auto-generated report from BCEAweb},
  hidelinks,
  pdfcreator={LaTeX via pandoc}}

\title{Auto-generated report from BCEAweb}
\author{}
\date{\vspace{-2.5em}Version: 21 December, 2024}

\begin{document}
\maketitle

\hypertarget{economic-analysis}{%
\section{Economic Analysis}\label{economic-analysis}}

This section contains a summary of the economic evaluation.

\hypertarget{cost-effectiveness-analysis}{%
\subsection{Cost-effectiveness
analysis}\label{cost-effectiveness-analysis}}

This sub-section presents a summary table reporting basic economic
results as well as the optimal decision, given the selected
willingness-to-pay threshold \(k=25000\).

\begin{verbatim}
  
  Cost-effectiveness analysis summary 
  
  Reference intervention:  Intervention1
  Comparator intervention(s): Intervention2
                            : Intervention3
                            : Intervention4
  
  Optimal decision: choose Intervention1 for k < 200
                           Intervention2 for 200 <= k < 300
                           Intervention4 for k >= 300
  
  
  Analysis for willingness to pay parameter k = 25000
  
                Expected net benefit
  Intervention1                10263
  Intervention2                17423
  Intervention3                22289
  Intervention4                28182
  
                                      EIB  CEAC   ICER
  Intervention1 vs Intervention2  -7160.3 0.056 158.66
  Intervention1 vs Intervention3 -12026.5 0.006 195.77
  Intervention1 vs Intervention4 -17919.7 0.000 198.33
  
  Optimal intervention (max expected net benefit) for k = 25000: Intervention4
             
  EVPI 1545.1
\end{verbatim}

\hypertarget{cost-effectiveness-plane}{%
\subsection{Cost-effectiveness plane}\label{cost-effectiveness-plane}}

The following graph shows the cost-effectiveness plane. This presents
the joint distribution of the population average benefit and cost
differential, \((\Delta_e,\Delta_c)\).

Each point in the graph represents a `potential future' in terms of
expected incremental economic outcomes. The shaded portion of the plane
is the `\emph{sustainability area}'. The more points lay in the
sustainability area, the more likely that the reference intervention
will turn out to be cost-effective, at a given willingness to pay
threshold, \(k\) (in this case selected at \(k=\) 25000)

\begin{center}\includegraphics{report_files/figure-latex/unnamed-chunk-28-1} \end{center}

\hypertarget{expected-incremental-benefit}{%
\subsection{Expected Incremental
Benefit}\label{expected-incremental-benefit}}

The following graph shows the Expected Incremental Benefit (EIB), as a
function of a grid of values for the willingness to pay \(k\) (in this
case in the interval 0 - 50000).

The value for \(k\) in correspondence of which the line crosses the
\(x-\)axis is termed the `\emph{break-even point}' and represents the
point(s) at which the optimal decision changes. The graph also reports
the 95\% credible limits around the EIB.

\begin{center}\includegraphics{report_files/figure-latex/unnamed-chunk-33-1} \end{center}

\hypertarget{cost-effectiveness-efficiency-frontier}{%
\subsection{Cost-effectiveness efficiency
frontier}\label{cost-effectiveness-efficiency-frontier}}

\begin{verbatim}
  
  Cost-effectiveness efficiency frontier summary 
  
  Interventions on the efficiency frontier:
                Effectiveness   Costs Increase slope Increase angle
  Intervention2       0.28824  45.733         158.66         1.5645
  Intervention4       0.72252 143.301         224.67         1.5663
  
  Interventions not on the efficiency frontier:
                Effectiveness  Costs     Dominance type
  Intervention1       0.48486 94.919 Extended dominance
  Intervention3       0.00000  0.000 Extended dominance
\end{verbatim}

\begin{center}\includegraphics{report_files/figure-latex/unnamed-chunk-38-1} \end{center}

\end{document}
